\begin{enumerate}
  \item Utredningen föreslår att alla lärare ska delta i 
    undervisningsverksamhet varje år,
    men den utelämnar den andra sidan av myntet: att alla lärare också bör 
    delta i aktivt forskningsskapande varje år.
    Detta påverkar uppfyllnaden av KTH:s mål att forskning och utbildning ska 
    gå hand-i-hand.
  \item Att ha möjlighet till befordran till lektor för adjunkter bara under en 
    tidsbegränsad period är inte en långsiktig lösning.
    Den rimmar dessutom väldigt dåligt med lika villkor.
    Att balansera forskning och undervisning är svårt.
    Disputerade adjunkter med 100\% undervisning behöver nå upp till samma krav 
    som en biträdande lektor med 25\% undervisning.
    Jag förstår resonemanget bakom en övergångsperiod,
    men trycker på att adjunkterna---som är lärare---bör permanent kunna ansöka 
    om befordran till lektor (med eller utan rätt till vidare prövning).
    Forskare bör behandlas separat, p.g.a.\ de problem om GRU-finansiering som 
    påtalas i utredningen.
    Att kunna ansöka om befordran till lektor utan rätt till prövning är bättre 
    än att inte kunna ansöka om befordran alls.
  \item Undervisningsintresserade (forskningsfokuserade) professorer har svårt 
    att finna tid för att utveckla sin undervisning.
    Forskningsintresserade (undervisningsfokuserade) adjunkter har, föga 
    förvånande, svårt att finna tid för att utveckla sin forskning.
    Detta är inte bra för KTH:s mål om att forskning och utbildning ska gå 
    hand-i-hand.
  \item Karriärsystemet är normerande, vilket motverkar målen om mångfald, 
    jämställdhet och lika villkor.
    Detta kan medföra att KTH inte når upp till sin fulla potential när det 
    gäller mångfald och således forskningens genomslagskraft.
\end{enumerate}
