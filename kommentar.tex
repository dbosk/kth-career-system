\documentclass[a4paper,oneside,article,swedish]{memoir}
\let\subsubsection\subsection
\let\subsection\section
\let\section\chapter
\usepackage[utf8]{inputenc}
\usepackage[T1]{fontenc}
\usepackage[swedish]{babel}

\usepackage[style=verbose,citestyle=verbose]{biblatex}
\addbibresource{kommentar.bib}

\usepackage{csquotes}
\usepackage{pdfpages}
\includepdfset{pages=-}
\usepackage{graphicx}
\usepackage{enumitem}
\usepackage{amsmath}

\usepackage{pgfplots}
\usepackage{pgf-pie}
\pgfplotsset{compat=1.17}
\usepackage{subcaption}

\usepackage[hidelinks]{hyperref}
\usepackage{cleveref}
\usepackage{didactic}

\title{Kommentar på utredningen om karriärsystemet}
\author{%
  Daniel Bosk, adjunkt\\EECS/TCS\\\texttt{dbosk@kth.se}
}

\begin{document}
\maketitle
\begin{abstract}
  Vi anser att förslaget är ett steg i rätt riktning, men att det inte går 
  tillräckligt långt.
  Vi anser att förslaget i utredningen har några tillkortakommanden.
  Vi anser följande:
  \begin{enumerate}
  \item Utredningen föreslår att alla lärare ska delta i 
    undervisningsverksamhet varje år,
    men den utelämnar den andra sidan av myntet: att alla lärare också bör 
    delta i aktivt forskningsskapande varje år.
    Detta påverkar uppfyllnaden av KTH:s mål att forskning och utbildning ska 
    gå hand-i-hand.
  \item Att ha möjlighet till befordran till lektor för adjunkter bara under en 
    tidsbegränsad period är inte en långsiktig lösning.
    Den rimmar dessutom väldigt dåligt med lika villkor.
    Att balansera forskning och undervisning är svårt.
    Disputerade adjunkter med 100\% undervisning behöver nå upp till samma krav 
    som en biträdande lektor med 25\% undervisning.
    Jag förstår resonemanget bakom en övergångsperiod,
    men trycker på att adjunkterna---som är lärare---bör permanent kunna ansöka 
    om befordran till lektor (med eller utan rätt till vidare prövning).
    Forskare bör behandlas separat, p.g.a.\ de problem om GRU-finansiering som 
    påtalas i utredningen.
    Att kunna ansöka om befordran till lektor utan rätt till prövning är bättre 
    än att inte kunna ansöka om befordran alls.
  \item Undervisningsintresserade (forskningsfokuserade) professorer har svårt 
    att finna tid för att utveckla sin undervisning.
    Forskningsintresserade (undervisningsfokuserade) adjunkter har, föga 
    förvånande, svårt att finna tid för att utveckla sin forskning.
    Detta är inte bra för KTH:s mål om att forskning och utbildning ska gå 
    hand-i-hand.
  \item Karriärsystemet är normerande, vilket motverkar målen om mångfald, 
    jämställdhet och lika villkor.
    Detta kan medföra att KTH inte når upp till sin fulla potential när det 
    gäller mångfald och således forskningens genomslagskraft.
\end{enumerate}

\end{abstract}
\clearpage
\tableofcontents*
\clearpage


\section{Inledning}

Det är bra att lärare utanför tenure track, får möjlighet att befordras.
Tyvärr går förslaget inte hela vägen.

Utredningen\autocite{utredning} sammanfattar KTH:s mål som följer:
\blockcquote[s.~2]{utredning}{%
  KTH ska ta ledningen för en hållbar samhällsutveckling, en utbildning av 
  högsta kvalitet och internationellt konkurrenskraftig och en forskning som 
  är internationellt ledande och som har genomslagskraft.
  Vidare ska KTH vara utvecklingsinriktat och kännetecknas av jämställdhet, 
  mångfald och lika villkor.
  Utbildning och forskning ska ha tätt samband där lärare både utbildar och 
  forskar.
  KTH ska vara attraktivt även ur internationellt perspektiv och det ska 
  finnas goda möjligheter till karriär- och kompetensutveckling.%
}

De föreslår följande:
\blockcquote[s.~2]{utredning}{%
  Med utgångspunkt i ovanstående ställningstaganden vill vi betona 
  läraranställningens
  särskilda plats på ett universitet. Vi menar, med stöd i KTH:s 
  anställningsordning och
  rekryteringsstrategi, att rekrytering i första hand ska göras till 
  läraranställningarna
  biträdande lektor, lektor och professor. Vi menar också att vi även 
  fortsättningsvis har behov
  av att rekrytera forskare för uppgifter inom forskning. Därtill ser vi 
  fortsatta behov av att
  rekrytera adjunkter givet det utbildningsutbud som KTH för närvarande har.

  Vi föreslår att det inrättas en befattning som lektor utanför den existerande 
  karriärstegen, det
  vill säga en lektor som har varken rätt eller möjlighet att prövas för 
  befordran till professor.
  Bedömningen är att denna anställning fyller ett behov i organisationen 
  eftersom det finns
  både forskare som i dag fungerar som skugglektorer och adjunkter som är 
  disputerade. Under
  en övergångsperiod om ca 5 år föreslår vi att disputerade adjunkter och vissa 
  forskare ska
  kunna prövas för befordran till lektor utanför karriärstegen. Reglerna för 
  dessa två processer
  kommer vara olika men i båda fallen är det av största vikt att instruktioner, 
  strategiska
  ställningstaganden, urval och bedömningar är tydliga och transparenta så att 
  processerna får
  en hög legitimitet.%
%
%  \textelp{}
%
%  När det gäller basfinansiering föreslår vi att alla lärare inom karriärstegen 
%  har en viss
%  basfinansiering genom KTH:s direkta statsanslag, vilket dels tydliggör KTH:s 
%  förväntan på
%  medarbetaren, dels möjliggör kompetensutveckling, sakkunniguppdrag, 
%  samverkan, kollegialt
%  arbete, arbete med forskarutbildning och annat som åligger en 
%  universitetsanställd lärare men
%  som inte alltid kan konteras på vare sig forskningsprojekt eller 
%  grundutbildning. Om den nya
%  lektorsanställningen införs finns det skäl att överväga om även dessa 
%  lektorer, utanför
%  karriärstegen, ska ha stabil basfinansiering.%
}
Vi kommer nu att gå igenom de givna målen och hur vi anser att utredningens 
förslag inte uppfyller dessa.

%Förslaget att ge adjunkter möjlighet till befordran är begränsat till en period 
%på fem år. Detta är inte en långsiktig lösning.
%
%Adjunkter borde kunna permanent inkluderas i befordringsprocessen. Samma som 
%för biträdande lektor. Annars är det att underkänna sakkunniggranskningens 
%kvalitet vid utvärderingen av huruvida biträdande lektor kan befordras till 
%lektor. Begränsningen på fem år är dessutom kort för adjunkter. En adjunkt har 
%normalt 80\% undervisning. En biträdande lektor har 25\% undervisning på en 
%period på upp till sex år och förväntas under denna period uppnå kraven för 
%lektor. För en adjunkt är detta en mycket längre process.
%
%\blockcquote[s.~2]{direktiv}{%
%  KTH:s tenure track-system är ett viktigt instrument för att kunna attrahera 
%  internationellt framstående forskare tidigt i karriären.
%  Där kan KTH vara mer internationellt konkurrenskraftig som arbetsgivare än 
%  när det gäller rekrytering av seniora lärare och forskare.
%  Tenure track-systemet består av biträdande lektor, lektor och professor, där 
%  den huvudsakliga rekryteringen förväntas ske till biträdande lektor.
%  De biträdande lektorerna har enligt högskoleförordningen rätt att prövas för 
%  befordran till lektorer och ska ges förutsättningar att utveckla sina 
%  kvalifikationer till lektor.
%  Den absoluta majoriteten av alla biträdande lektorer på KTH befordras till 
%  lektorer.%
%}
%
%Jag kommer i huvudsak att kommentera förslaget att
%\blockcquote[s.~2]{utredning}{%
%  det inrättas en befattning som lektor utanför den existerande 
%  karriärstegen, det vill säga en lektor som har varken rätt eller möjlighet 
%  att prövas för befordran till professor.%
%}
%
%\blockcquote[s.~2]{utredning}{%
%  Bedömningen är att denna anställning fyller ett behov i organisationen 
%  eftersom det finns både forskare som i dag fungerar som skugglektorer och 
%  adjunkter som är disputerade.
%  Under en övergångsperiod om ca 5 år föreslår vi att disputerade adjunkter 
%  och vissa forskare ska kunna prövas för befordran till lektor utanför 
%  karriärstegen.
%  Reglerna för dessa två processer kommer vara olika men i båda fallen är det 
%  av största vikt att instruktioner, strategiska ställningstaganden, urval 
%  och bedömningar är tydliga och transparenta så att processerna får en 
%  hög legitimitet.
%}%

\section{En konflikt mellan mål}\label{KonfliktFroskningUndervisning}

Direktivet för utredningen säger att
\blockcquote[s.~1]{direktiv}{%
  KTH ska ha ett tätt samband mellan utbildning och forskning, där lärare både 
  undervisar och forskar.%
}
Utredningen har inte missat detta, \textcite{utredning} skriver att
\blockcquote[s.~2]{utredning}{%
  \textins{u}tbildning och forskning ska ha tätt samband där lärare både 
  utbildar och forskar%
}
och föreslår därför att därför
\blockcquote[s-~2]{utredning}{%
  att alla med läraranställning på KTH någon gång under ett läsår ska möta 
  studenter i någon form av utbildningsinsats.%
}

Jag skulle då tycka att det vore naturligt att även lägga till
\blockquote{%
  att alla med läraranställning på KTH någon gång under året ska delta i någon 
  form av forskningsskapande verksamhet.%
}
Att passivt ta till sig annans forskning är inte vad som avses---det bör alla 
göra ändå---de ska delta i skapandet.
% Tack Alexander för RAE!
Detta stämmer även utvärderarna i RAE 2021 med om:
\blockcquote[s.~19]{RAE2021}{\foreignlanguage{english}{%
  it was observed
  that some departments recruit lecturers whom are supposed
  to focus entirely on teaching which, according to the evaluators, is 
  questionable for such a strong research department,
  since it seems to go against the university ambition that
  \enquote{teachers are researchers and researchers are teachers}.%
}}
Området för forskningen de deltar i bör vara relaterat till deras undervisning, 
men i linje med den akademiska friheten bör detta lämnas till den enskilde---må 
det vara forskning inom undervisningsämnet eller pedagogisk forskning om sin 
undervisning.
Ett vetenskapligt förhållningssätt är centralt för examina på alla nivåer, 
redan från högskoleexamen.\footnote{%
  Högskoleexamen är en tvåårig examen, alltså ett år kortare än en 
  kandidatexamen.
}\autocite[bilaga 2]{Högskoleförordningen}
Det finns ett mervärde för studenten att se att läraren även har ett 
vetenskapligt förhållningssätt till sin undervisning---inte bara 
ämnesområdet---nämligen att detta fördjupar studenternas förståelse för 
vetenskaplighet och forskning generellt\autocite{NCOL}.

Med tanke på detta borde även visst basanslag åsidosättas för att adjunkter ska 
kunna aktivt delta i forskningverksamhet.


\section{Ojämlika villkor och 
ojämlikhet}\label{ForskandeAdjunkter}\label{OjämlikaVillkor}

Anställningsordningen\autocite{Anställningsordning} ställer upp kraven för en 
lektor.
En adjunkt har inga som helst problem att uppnå de pedagogiska kraven ställda 
på lektorer.
Det som måste tas i beaktning är de vetenskapliga bedömningsgrunderna, då 
vetenskaplig verksamhet inte är huvudsaklig del av adjunkters 
tjänstebeskrivning.
Anställningsordningen skriver att
\blockcquote[s.~20, min emfas]{Anställningsordning}{%
  \textins{a}rbetsuppgifter inom forskning och utvecklingsarbete \emph{kan} 
  inbegripa forskning inom en forskargrupp samt pedagogiskt och ämnesmässigt 
  utvecklingsarbete%
}.
Så anställningsordningen förhindrar inte att adjunkter deltar i 
forskningsverksamheten.
Den anger följande för vetenskapliga bedömningsgrunder för lektorer, som en 
adjunkt då måste uppfylla vid befordran:
\blockcquote[ss.~11--12]{Anställningsordning}{%
  Den sökande bör visa vetenskaplig skicklighet genom att
  \begin{itemize}
    \item uppvisa förmåga att självständigt formulera och lösa forskningsproblem.
    \item vara verksam inom, eller nära, den internationella forskningsfronten.
    \item ha publicerat vetenskapliga arbeten av hög kvalitet i icke ringa 
      omfattning i internationellt erkända vetenskapliga tidskrifter eller andra 
      publikationsformer som är aktuella inom det specifika ämnesområdet.
    \item uppvisa förmåga att leda forskningsverksamhet. Sådan förmåga kan visas 
      genom exempelvis dokumenterad erfarenhet av projektledning i 
      forskningsprojekt, handledning i utbildning på forskarnivå, handledning av 
      postdoktorer eller andra relevanta ledningsuppdrag.
    \item uppvisa förmåga att söka, erhålla medel för och driva forskningsprojekt 
      utifrån exempelvis publikationer, rapporter och beslut om beviljade anslag.
  \end{itemize}
  Vid bedömningen bör sökande visa skicklighet inom alla bedömningsgrunder och 
  graden av skicklighet inom respektive bedömningsgrund ska utvärderas. En samlad 
  bedömning ska göras av sökandes vetenskapliga skicklighet.

  Vidare ska den sökande visa
  \begin{itemize}
    \item förmågan att samverka med det omgivande samhället för ömsesidigt utbyte 
      och ha verkat för att den kunskap och kompetens som finns vid universitetet 
      kommer samhället till nytta.
    \item skicklighet att utveckla och leda verksamhet och personal. Däri ingår 
      att ha kunskap om mångfalds- och likabehandlingsfrågor med särskilt fokus 
      på jämställdhet.
    \item administrativ skicklighet.
    \item samarbetsförmåga.
  \end{itemize}
  I anställningsprofilen för det aktuella anställningsärendet ska det anges 
  vilken betydelse respektive bedömningsgrund har i förhållande till varandra. I 
  anställningsprofilen får ytterligare bedömningsgrunder för anställningen 
  fastställas.%
}

En biträdande lektor har brukligt omkring 25\% undervisning och 75\% forskning.
Detta för att
\blockcquote[kap.~4, 12 a §]{Högskoleförordningen}{%
  ges möjlighet att utveckla sin självständighet som forskare och meritera sig 
  såväl vetenskapligt som pedagogiskt för att uppfylla kraven på behörighet för 
  en anställning som lektor%
} under en tid på
\blockcquote[kap.~4, 12 a §]{Högskoleförordningen}{%
  minst fyra och högst sex år%
}.
Det ger en biträdande lektor 3--4.5 heltidsforskningsår att uppfylla kraven för 
lektor ovan.

En adjunkt har brukligt omkring 80\% undervisning och 20\% kompetensutveckling.
Låt oss säga att adjunkten kan använda all sin kompetensutveckling till 
forskning\footnote{%
  Egentligen binds den upp på annat, så detta är lite väl optimistiskt.
  Alla lärare får ju också motsvarande kompetensutveckling för den del de 
  undervisar.
  Det är bara att adjunkter undervisar normalt 100\%, vilket ger 20\% 
  kompetensutveckling.
}.
Så under de sex åren som den biträdande lektorn har 4.5 forskningsår på sig att 
uppfylla kraven för lektor, har adjunkten 1.2 heltidsforskningsår.

Dessa siffror understryker också hur jämställda undervisningen och forskningen 
är:
för att kvalificera sig som lektor krävs 4.5 forskningsår, men bara 1.5 
undervisningsår.
Det är vad en biträdande lektor, som vid rekryteringen har mer erfarenhet av 
forskning än undervisning, uppskattas behöva för att uppfylla kraven på 
pedagogisk skicklighet.

Att då ha en befordringsprocess som är begränsad till
\blockcquote[s.~2]{utredning}{%
  en övergångsperiod om ca 5 år%
} är inte rimligt.
Den bör vara permanent.
Jag säger inte att adjunkter ska få 75\% av tjänsten avsatt för forskning, jag 
menar att det måste få ta den tid de har möjlighet att lägga på forskning.
Samma behörighetskrav ska gälla, då måste båda få de 4.5 forskningsår som 
krävs---men utspridda över en längre tid.
Biträdande lektorerna får tiden mer koncentrerat, adjunkterna bör få den mer 
utspridd utefter vad som är rimligt i deras tjänst.
Men det är fullt möjligt för en adjunkt att kvalificera sig som lektor och om 
så är fallet bör adjunkten kunna bli befordrad till lektor\footnote{%
  Vilket är fallet på många andra lärosäten.
}.

Vi har förståelse att denna övergångsperiod är en konsekvens av att adjunkter 
och forskare behandlas tillsammans.
Vi anser att adjunkter, som är lärare, bör hanteras separat från forskare och 
på samma sätt som övriga lärare, med rätt att ansöka om befordran (med eller 
utan rätt till vidare prövning).
Det är enbart forskarnas icke-lärartjänst som behöver hanteras med en 
övergångsperiod för att hantera att de görs om till lärartjänst (lektor).


\section{Att balansera flera forskningsämnen}

Adjunkter är, som uttrycks i
anställningsordningen\autocite[avsnitt 1.5, ss.~20]{Anställningsordning},
huvudsakligen befattade med undervisning.
Många är pedagogiskt intresserade och bedriver viss pedagogisk forskning.
Det är tidsmässigt tacksamt att forska på det man gör mest.
Att hinna med både ämnesmässig och pedagogisk forskning är dock svårt.
Tiden är begränsad och att ägna sig åt den ena sker på bekostnad av den andra.
Detta är något som måste tas i beaktande i kraven för befordring till lektor 
för adjunkter, så att både ämnesmässig och pedagogisk forskning värderas och 
vägs tillsammans.


\section{Ett normerande karriärsystem}

För att problematisera detta,
låt oss undersöka en aspekt av mångfald bland studenterna på några av våra 
program.

\begin{table}
  \begin{sidecaption}[Åldersfördelning av studenter på olika program]{%
    Antal sökande och antagna per program samt åldersgrupp.
    Statistiken är hämtad från UHR:s antagningsstatistik för hösten 2024 
    \parencite{UHRstat}.
    De program med flest antal sökande över 34 år är ortsoberoende, övriga 
    program är campusprogram.
  }[ÅlderStudenter]
  \flushright
  \begin{tabular}{lrrr}
    \toprule
    & \textbf{Under 25} & \textbf{25--34} & \textbf{Över 34} \\
    \midrule
    \textbf{Sökande KTH indek} & 2873 & 151 & 26 \\
    \textbf{Antagna KTH indek} & 201 & 8 & 1 \\
    \midrule
    \textbf{Sökande KTH maskin} & 1906 & 126 & 27 \\
    \textbf{Antagna KTH maskin} & 184 & 4 & 2 \\
    \midrule
    \textbf{Sökande LTU kand} & 448 & 928 & 536 \\
    \textbf{Antagna LTU kand} & 9 & 24 & 12 \\
    \midrule
    \textbf{Sökande LTU master} & 34 & 236 & 266 \\
    \textbf{Antagna LTU master} & 4 & 25 & 32 \\
    \midrule
    \textbf{Sökande LTU indek} & 434 & 22 & 3 \\
    \textbf{Antagna LTU indek} & 83 & 2 & 0 \\
    \bottomrule
  \end{tabular}
  \end{sidecaption}
\end{table}

\begin{figure}
  \begin{sidecaption}[Procentuell åldersfördelning av studenter på olika 
    program]{%
    Procentuell åldersfördelning av studenter på olika program.
    Statistiken är hämtad från UHR:s antagningsstatistik för hösten 2024 
    \parencite{UHRstat}.
    De program med flest antal sökande över 34 år är ortsoberoende, övriga 
    program är campusprogram.
  }[fig:age_distribution_normalized]
  \centering
  \begin{tikzpicture}
    \begin{axis}[
      ybar stacked,
      bar width=7pt,
      width=\linewidth,
      height=7cm,
      enlargelimits=0.15,
      ylabel={Andel},
      xlabel={Program},
      symbolic x coords={KTH indek, KTH maskin, LTU kand, LTU master, LTU indek},
      xtick=data,
      xticklabel style={rotate=45, anchor=east},
      legend cell align={left},
      legend style={at={(1.02,0.3)}, anchor=north west},
      yticklabel={\pgfmathprintnumber{\tick}\%}
    ]
      % Procentuell fördelning per åldersgrupp per program
      \addplot+[bar shift=-7pt, fill=blue] plot coordinates {(KTH indek,94.1) (KTH maskin,92.6) (LTU kand,23.4) (LTU master,6.3) (LTU indek,94.5)};
      \addplot+[bar shift=-7pt, fill=blue!50] plot coordinates {(KTH indek,5.0) (KTH maskin,6.1) (LTU kand,48.5) (LTU master,44.0) (LTU indek,4.8)};
      \addplot+[bar shift=-7pt, fill=blue!20] plot coordinates {(KTH indek,0.9) (KTH maskin,1.3) (LTU kand,28.1) (LTU master,49.6) (LTU indek,0.7)};
      \legend{Sökande under 25, Sökande 25--34, Sökande över 34}
    \end{axis}
  \end{tikzpicture}
  \end{sidecaption}
\end{figure}

Vi ser antalet sökande och antalet antagna i olika åldersgrupper i 
\cref{ÅlderStudenter} (visualiserat i \cref{fig:age_distribution_normalized}).
Vi kontrasterar studenterna på civilingenjörsprogrammen i industriell ekonomi 
(indek) och maskinteknik (maskin) vid KTH\footnote{%
  Jag ger den inledande programmeringskursen för dessa program, en av de första 
  kurser de läser, så jag har observerat studentsammansättningen.
} med studenterna på kandidatprogrammet i systemvetenskap, masterprogrammet i 
data science och civilingejörsprogrammet i industriell ekonomi vid 
LTU\footnote{%
  Jag valde LTU då de har ett civilingejörsprogram i indek samt ortsoberoende 
  program på både kandidat- och masternivå.
  Kandidatprogrammet i systemvetenskap är det program med flest, i antal, 
  sökande över 34 år inför höstterminen 2024, nationellt bland program med 
  datainriktning.
}.

Vi kan tydligt se att några program har en väsentligen större andel sökande 
över 34 år än andra (båda vid LTU).
Dessa program har studieorten \enquote{ortsoberoende} och ges alltså på 
distans.
Som kontrast har vi civilingejörsprogrammet i indek vid LTU, som följer samma 
mönster som mostsvarande program vid KTH---så nej, det är inte Luleå som lockar 
äldre mer än Stockholm, det är undervisningsformen.
Trenden i datat reflekterar även min erfarenhet från när jag undervisade på 
distansprogram vid Mittuniversitetet\footnote{%
  Jag undervisade på distansprogrammet Nätverksdrift, som hösten 2024 hade 83 
  sökande under 25, 171 sökande 25--34 och 135 sökande över 34 år.
  Jag gav även en kurs i informationssäkerhet för ett masterprogram i 
  informatik.
}.
På distansprogrammen är det större mångfald bland studenterna än på 
campusprogrammen:
med åldern på studenterna kommer större mångfald av erfarenheter.
Fler har jobbat tidigare, vissa inom flertalet olika områden.
Flera jobbar samtidigt, har familj, bolån, etc.
Men det förekommer också några gymnasieungdomar.
Medan det på campusprogrammen förekommer i princip bara gymnasieungdomar.
Denna mångfald är fördelaktig för gymnasieungdomarnas utveckling, den går 
betydligt snabbare då.
Exempelvis:
I en klass med gymnasieelever måste man i princip dra frågorna ur dem.
I en klass med äldre studenter, som sedan länge har passerat stadiet 
\enquote{att det är pinsamt att ställa dumma frågor}, har inga problem med att 
ställa frågor eller ta initiativ till diskussioner.
De äldre studenterna kan även bidra med andra perspektiv, såsom erfarenheter 
från arbetslivet.

\Textcite{DiverseLabs} skriver att
\blockcquote[%
  Sammanfattat av o3-mini:
  \texttt{pdf2txt ./nature-diverse-teams.pdf | sc \enquote{Sammanfatta vad det 
  är som gör att forskarlag med mångfald är bättre.}}
]{DiverseLabs}{%
  Forskargrupper med mångfald visar sig ofta vara mer framgångsrika eftersom de 
  erbjuder en kombination av olika perspektiv, erfarenheter och kompetenser. 
  Genom att samla människor med varierande etnicitet, kön, nationellt ursprung, 
  ålder och vetenskaplig bakgrund kan teamet formulera nya, kreativa frågor och 
  lösningar. Detta leder till högre publiceringsfrekvens, fler citeringar och 
  en större relevans gentemot olika samhällsgrupper. Samtidigt bidrar mångfald 
  till en kultur av respekt och inkluderande samarbeten, där traditionell 
  kunskap – exempelvis den uråldriga maorikulturen – kan integreras med 
  västerländska metoder. Ett inkluderande synsätt hjälper inte bara till att 
  lösa samhällsproblem på ett mer träffsäkert sätt, utan stärker även labbens 
  interna dynamik och innovationsförmåga genom att utmana invanda 
  tankemönster.%
}
Fördelarna med mångfalden träder fram även i detta sammanhang.
Och det har naturligtvis inte undgått KTH:
texten i direktivet säger att
\blockcquote[s.~1]{direktiv}{%
  KTH \textins{ska} vara ett utvecklingsinriktat och internationellt 
  universitet som kännetecknas av jämställdhet, mångfald och lika villkor%
} samt att
\blockcquote[s.~1]{direktiv}{%
  KTH ska attrahera framstående lärare och forskare från hela världen%
}.
Det som däremot verkar ha undgått KTH\footnote{%
  Även riksdagen i viss mån, i sin utformning av lärarkategorin biträdande 
  lektor.
} är att: som man anställer får man anställda.
Direktivet säger att
\blockcquote[s.~2, min emfas]{direktiv}{%
  \textins{t}enure track-systemet består av biträdande lektor, lektor och 
  professor, där \emph{den huvudsakliga rekryteringen förväntas ske till 
  biträdande lektor.}%
}
En konsekvens av detta är att de enda som kan komma in i tenure track-systemet 
är de
\blockcquote[4 kap, 4 a §]{Högskoleförordningen}{%
  som har avlagt doktorsexamen eller har nått motsvarande kompetens högst fem 
  eller högst sju år innan tiden för ansökan av anställningen som biträdande 
  lektor har gått ut%
%  Även den som har avlagt doktorsexamen eller har uppnått motsvarande kompetens 
%  tidigare kan dock komma i fråga om det finns särskilda skäl. Med särskilda 
%  skäl avses ledighet på grund av sjukdom, föräldraledighet eller andra 
%  liknande omständigheter.%
} samt kan ta en anställning som är tidsbegränsad till
\blockcquote[4 kap, 12 a §]{Högskoleförordningen}{%
  minst fyra och högst sex år%
} innan den blir tillsvidare.
Detta skulle jag argumentera är en tämligen snäv rekryteringsbas.

Låt oss börja med att titta på doktoranderna.
Enligt \textcite{UKÄstat} är \emph{median}åldern för doktorander på KTH 29 år 
(HT2023), se \cref{ÅldersfördelningDoktorander}.
Medianåldern för doktorander nationellt har varit stabilt, det har pendlat 
mellan 32 och 33 år över åren från 1999 till 2023.
Cirka 40\% av KTH:s rekryterade doktorander kommer från KTH:s egen  
grundutbildning\autocite{UKÄstat}.
(Det var 42\% nationellt.)
Majoriteten av doktoranderna verkar alltså påbörja sin forskarutbildning kort 
efter sin grundutbildning (jämför \cref{ÅlderStudenter}).

\begin{table}
  \begin{sidecaption}{%
    Antal doktorander per åldersgrupp på KTH.
    Statistiken är hämtad från UKÄ:s statistik för höstterminen 2023 
    \parencite{UKÄstat}.
    Medianåldern för doktorander på KTH är 29 år.
    Notera att detta är fördelningen för alla aktiva doktorander vid KTH hösten 
    2023, \emph{inte} åldern för antagning som i \cref{ÅlderStudenter}.
  }%
  \label{ÅldersfördelningDoktorander}
  \hspace*{\fill}
  \begin{subfigure}[b]{0.45\textwidth}
    \centering
    \begin{tabular}{lrr}
      \toprule
      Kön & Åldersgrupp & Antal \\
      \midrule
      Kvinnor & -29    & 227 \\
      Kvinnor & 30-39  & 158 \\
      Kvinnor & 40-    & 26  \\
      \midrule
      Kvinnor & Total  & 411 \\
      \midrule
      Män     & -29    & 426 \\
      Män     & 30-39  & 298 \\
      Män     & 40-    & 25  \\
      \midrule
      Män     & Total  & 749 \\
      \midrule
      Total   & -29    & 653 \\
      Total   & 30-39  & 456 \\
      Total   & 40-    & 51  \\
      \midrule
      Total   & Total  & 1160 \\
      \bottomrule
    \end{tabular}
    \caption{Antal doktorander per åldersgrupp på KTH.}
    \label{ÅlderDoktorander}
  \end{subfigure}
  \hfill
  \begin{subfigure}[b]{0.45\textwidth}
    \centering
    \begin{tikzpicture}
      \pie[text=legend, radius=2, color={blue!50, red!50, green!50}]{56.3/-29, 39.3/30-39, 4.4/40-}
    \end{tikzpicture}
    \caption{Procentuell fördelning av doktorander per åldersgrupp på KTH, motsvarar total i \cref{ÅlderDoktorander}.}
    \label{ProcentuellÅlderDoktorander}
  \end{subfigure}
  \hspace*{\fill}
  \end{sidecaption}
\end{table}

Mångfalden hos de biträdande lektorerna skulle jag då säga utgörs endast av 
etnicitet\footnote{%
  Erfarenhet från andra forskargrupper bidrar kanske inte med så mycket 
  mångfald, då grupper inom samma har en tendens att samarbeta.
  Men viss mångfald bidrar det ändå med för arbetsmiljön.
}---endast \emph{ett} av de attribut som listades ovan.
Erfarenheten som samlas under dessa maximalt sju år är sannolikt av akademisk 
karaktär, således tämligen homogen.

Om man däremot samlat på sig andra erfarenheter, exempelvis inom industrin, då 
är det svårare att komma tillbaka in i akademin.
Det är i princip bara tjänsterna adjunkt och forskare\footnote{%
  Det finns ju även möjlighet som adjungerad professor eller lektor.
  Även rekrytering direkt som lektor eller professor, men dessa skulle användas 
  sparsamt och det kan vara svårt att samla på sig de akademiska 
  kvalifikationer som man behöver för att komma in på dessa steg.
} som finns att tillgå, vilket ytterligare talar för att inkludera adjunkterna 
i forskningsverksamheten (\cref{ForskandeAdjunkter}).

Precis som med studenterna, där valet av undervisnings form (campus eller 
distans) påverkade studentsammansättningen, påverkar möjligheterna i 
karriärsystemet vilka som anställs.
Karriärsystemets utformning har en normerande effekt som bör tas i 
beaktande.
Precis som vi uppmanas till normkritik i undervisningen\footnote{%
  Se examinatorskursen.
}, så bör vi även vara normkritiska när vi utformar vårt karriärsystem.
Den normerande effekten kan göra att vi inte uppfyller våra mål med 
mångfald, lika villkor och jämställdhet---åtminstone inte så väl som vi 
faktiskt skulle kunna uppfylla dem.
De ansökningar man aldrig får, kommer man aldrig att se.

%Exempelvis PriSec-gruppen vid Karlstads universitet har som princip att inte 
%anställa sina doktorander som post-doktorer eller biträdande lektorer efter 
%disputation för att förhindra \enquote{akademisk inavel}.

\section{Avslutning}

Sammanfattningsvis anser vi att förslaget är ett steg i rätt riktning, men att 
det inte går hela vägen.
\begin{enumerate}
  \item Utredningen föreslår att alla lärare ska delta i 
    undervisningsverksamhet varje år,
    men den utelämnar den andra sidan av myntet: att alla lärare också bör 
    delta i aktivt forskningsskapande varje år.
    Detta påverkar uppfyllnaden av KTH:s mål att forskning och utbildning ska 
    gå hand-i-hand.
  \item Att ha möjlighet till befordran till lektor för adjunkter bara under en 
    tidsbegränsad period är inte en långsiktig lösning.
    Den rimmar dessutom väldigt dåligt med lika villkor.
    Att balansera forskning och undervisning är svårt.
    Disputerade adjunkter med 100\% undervisning behöver nå upp till samma krav 
    som en biträdande lektor med 25\% undervisning.
    Jag förstår resonemanget bakom en övergångsperiod,
    men trycker på att adjunkterna---som är lärare---bör permanent kunna ansöka 
    om befordran till lektor (med eller utan rätt till vidare prövning).
    Forskare bör behandlas separat, p.g.a.\ de problem om GRU-finansiering som 
    påtalas i utredningen.
    Att kunna ansöka om befordran till lektor utan rätt till prövning är bättre 
    än att inte kunna ansöka om befordran alls.
  \item Undervisningsintresserade (forskningsfokuserade) professorer har svårt 
    att finna tid för att utveckla sin undervisning.
    Forskningsintresserade (undervisningsfokuserade) adjunkter har, föga 
    förvånande, svårt att finna tid för att utveckla sin forskning.
    Detta är inte bra för KTH:s mål om att forskning och utbildning ska gå 
    hand-i-hand.
  \item Karriärsystemet är normerande, vilket motverkar målen om mångfald, 
    jämställdhet och lika villkor.
    Detta kan medföra att KTH inte når upp till sin fulla potential när det 
    gäller mångfald och således forskningens genomslagskraft.
\end{enumerate}

Vi förstår resonemanget bakom övergångsresonemanget,
men trycker på att adjunkterna, som är lärare, bör permanent kunna ansöka om 
befordran till lektor (med eller utan rätt till vidare prövning).
Att kunna ansöka om befordran till lektor utan rätt till prövning är bättre än 
att inte kunna ansöka om befordran alls.

\printbibliography
\end{document}
